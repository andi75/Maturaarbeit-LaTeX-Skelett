\documentclass[a4paper,11pt]{report}

\author{A.~Umbach}
\title{Mein Maturaarbeitsthema}

\usepackage[ngerman]{babel}
\usepackage[latin1]{inputenc}
\usepackage{blindtext}

%minimale page header & footer
\usepackage{fancyhdr}
\pagestyle{fancy}
\setlength{\headheight}{14pt} 
\fancyhf{}
\fancyhead[C]{\nouppercase{\leftmark}}
\fancyfoot[C]{\thepage}

%create links in the pdf
\usepackage[colorlinks]{hyperref}

%a glossary
\usepackage[toc]{glossaries}


\newglossaryentry{computer}
{
  name=computer,
  description={is a programmable machine that receives input,
               stores and manipulates data, and provides
               output in a useful format}
}
\newglossaryentry{blahblah}
{
  name=blahblah,
  description={\blindtext}
}

\makeglossaries



\begin{document}
%\maketitle
\tableofcontents

\chapter{Einleitung}
\section{Thema}
\blindtext
\chapter{Einf�hrung}
\section{Vorraussetzungen}
In dieser Arbeit setze ich ein wenig Vorwissen voraus. Dazu geh�rt....
\blindtext

\section{Grundlagen}
\subsection{Die ganz elementaren Sachen}
Hier wird's beschrieben. Was weniger elementar ist, steht in Abschnitt \ref{etwaswenigerElementar} auf Seite \pageref{etwaswenigerElementar}.
\blindtext
\subsection{Ein paar weniger elementare Sachen}
\label{etwaswenigerElementar}

\blindtext[2]



\subsection{Und noch das Glossar}
Die Ausdr�cke \gls{computer} und \gls{blahblah} kommen im Glossar vor.

\printglossaries


\end{document}